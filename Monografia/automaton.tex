\chapter{Autômato de sufixos} \label{chap:suffixautomaton}

Ao construir e usar uma árvore de sufixos de~$P$, temos que cada vértice representa uma substring de~$P$, e isso é uma noção bem intuitiva, que facilita usar essa estrutura para resolver problemas com strings. No entanto, o número de vértices pode ser quadrático em~$|P|$, e por isso é necessário algumas vezes comprimi-la, o que complica o código.

A estrutura de um autômato de sufixos, que veremos nesse capítulo, não envolve compressão de vértices, o que deixa o código mais simples, mas cada vértice representa algo menos intuitivo, ainda que bem útil.

\section{Ocorrências à direita}

Dizemos que~$i$ é uma ocorrência à direita de~$S$ em~$P$ se~$S \suff P[1\tdots i]$, ou seja,~$S$ ocorre em~$P$ começando na posição~$i - |S| + 1$.

\begin{definition}
Seja~$D_P(S) = \{i : S \suff P[1\tdots i] \}$ o conjunto das ocorrências à direita de~$S$ em~$P$. Se~$P$ estiver claro do contexto, escrevemos~$D(S)$ em vez de~$D_P(S)$. Se~${D_P(A) = D_P(B)}$, dizemos que~$A \equiv_P B$, ou seja,~$A$ e~$B$ estão na mesma classe de equivalência de ocorrências à direita. Da mesma forma, escrevemos~$A \equiv B$ se~$P$ está claro do contexto.
\end{definition}

É fácil notar que~$\equiv_P$ é uma relação de equivalência. Nesse capítulo, usaremos que~$AB$ é a concatenação das strings~$A$ e~$B$.

Vamos provar alguns fatos sobre ocorrências à direita que ajudam a entender as classes de equivalência de~$\equiv$, e as provas envolvidas na criação do algoritmo. Só trabalharemos com strings que são substrings de~$P$, então pode-se assumir que o conjunto~$D$ das strings analisadas é não-vazio.

\begin{lemma}
\label{lem:R1}
Se~$A \suff B$ então~$D(B) \subseteq D(A)$.
\end{lemma}

\begin{proof}
Seja~$i \in D(B)$ uma ocorrência à direita de~$B$. Então~$B \suff P[1\tdots i]$. Como~$A \suff B$, temos que~$A \suff P[1\tdots i]$ e~$i \in D(A)$.
\end{proof}

\begin{lemma}
\label{lem:R1.5}
Se~$D(A) \subseteq D(B)$ então~$A \suff B$ ou~$B \suff A$.
\end{lemma}

\begin{proof}
Se~$i \in D(A)$, então~$i \in D(B)$. Logo, neste caso,~$A \suff P[1\tdots i]$  e~$B \suff P[1\tdots i]$. Como~$A$ e~$B$ são sufixos da mesma string,~$A \suff B$ ou~$B \suff A$.
\end{proof}

\begin{corollary}
\label{col:R2}
Se~$A \equiv B$ e~$|A| \leq |B|$ então~$A \suff B$.
\end{corollary}

\begin{proof}
Como~$A \equiv B$, temos que~$D(A) = D(B)$ e, pelo Lema~\ref{lem:R1.5},~$A \suff B$ ou~$B \suff A$. O resultado segue de~$|A| \leq |B|$.
\end{proof}

Isso mostra que todas as strings de uma classe de equivalência estão relacionadas pois são sufixos umas das outras.

\begin{prop} \label{prop:classc}
Se~$A \equiv B$, e~$c \in \E$, então~$Ac \equiv Bc$.
\end{prop}

\begin{proof}
Note que para que~$i$ esteja em~$D(Ac)$ é necessário e suficiente que~${i-1 \in D(A)}$ e~${P[i] = c}$. Como~$D(A) = D(B)$, vale que~$D(Ac) = D(Bc)$ e assim~$Ac \equiv Bc$.
\end{proof}

Para todo~$A \neq \emptystring$, seja~$\stfail_P(A)$ a maior string~$B \suffp A$ tal que~$B \not\equiv_P A$. Note que~$A \not\equiv_P \emptystring$, logo~$\stfail_P(A)$ está bem definida. Dizemos que~$\stfail_P(A)$ é a \emph{string de falha} de~$A$ para~$P$. Se~$P$ está claro do contexto, escrevemos apenas~$\stfail(A)$.

\begin{lemma}
\label{lem:R4}
Se~$A \equiv B$ então~$\stfail(A) = \stfail(B)$.
\end{lemma}

\begin{proof}
Assuma sem perda de generalidade que~$|A| \leq |B|$. Então, pelo Corolário~\ref{col:R2}, temos que~$A \suff B$.
Como~$\stfail(A) \suffp A \suff B$ e~$\stfail(A) \not\equiv A \equiv B$, vale que~$|\stfail(B)| \geq|\stfail(A)|$.
Se~${A \suff \stfail(B)}$, pelo Lema~\ref{lem:R1}, temos~$D(\stfail(B)) \subseteq D(A) = D(B)$, uma contradição pois~$D(B) \subsetneq D(\stfail(B))$ pois~$\stfail(B) \suffp B$ e~$\stfail(B) \not\equiv B$.
Assim sendo,~$\stfail(B) \suffp A$ e, como~$A \not\equiv \stfail(B)$, vale que~$|\stfail(B)| \leq |\stfail(A)|$ e o lema segue.
\end{proof}

\begin{lemma}
\label{lem:R6}
Se~$A \neq \emptystring$, então~$\stfail(A)$ é a maior string de sua classe de equivalência.
\end{lemma}

\begin{proof}
Suponha que exista string~$B$ tal que~$B \equiv \stfail(A)$ e~$|B| > |\stfail(A)|$.
Como~${\stfail(A) \suffp A}$, o Lema~\ref{lem:R1} implica que~$D(A) \subset D(\stfail(A)) = D(B)$. Usando o Lema~\ref{lem:R1.5}, temos que~$A \suffp B$ ou~$B \suffp A$. Se~$A \suffp B$ então~$D(B) \subset D(A)$, uma contradição. Se~$B \suffp A$ então temos uma contradição pois~$\stfail(A)$ é o \emph{maior} sufixo de~$A$ de uma classe diferente.
\end{proof}

\begin{corollary}
\label{col:falha}
A string~$\stfail_P(P)$ é o maior sufixo de~$P$ que ocorre pelo menos duas vezes em~$P$.
\end{corollary}

\begin{proof}
Note que~$D(P) = \{|P|\}$, logo~$|D(\stfail(P))| > 1$, ou seja,~$\stfail(P)$ ocorre mais de uma vez em~$P$. Pelo Lema~\ref{lem:R6}, essa string é a maior de sua classe de equivalência.
\end{proof}

\begin{figure}
\centering
    \begin{tikzpicture}
\matrix[matrix of nodes, nodes in empty cells,
row 1/.style={nodes={draw, minimum size = 8mm}},
row 1 column 14/.style={nodes={draw=none,right}}] (array) {
 & & & & & & & & & & & & &~$B$  \\};
 
\draw[decoration={brace, raise=5pt},decorate]
  (array-1-7.north west) -- node[above=6pt] {$A$} (array-1-13.north east);

\draw[decoration={brace, raise=5pt, mirror},decorate]
  (array-1-5.south west) -- node[below=6pt] {$\stfail^i_P(B)$} (array-1-13.south east);
  
 
\draw[decoration={brace, raise=30pt, mirror},decorate]
  (array-1-9.south west) -- node[below=31pt] {$\stfail^{i+1}_P(B)$} (array-1-13.south east);
  
 
\end{tikzpicture}

    \caption{Explicação Lema~\ref{lem:multifail}.}
    \label{fig:faili}
\end{figure}

\begin{lemma} \label{lem:multifail}
Se~$A \suff B$, então existe~$i \geq 0$ tal que~$\stfail^i(B) \equiv A$, onde usamos que~${\stfail^0(B) = B}$ e~${\stfail^i(B) = \stfail(\stfail^{i-1}(B))}$.
\end{lemma}

\begin{proof}
Escolha~$i \geq 0$ máximo tal que~${|\stfail^i(B)| \geq |A|}$. Então claramente~${A \suff \stfail^i(B)}$. A Figura~\ref{fig:faili} ilustra tal situação. Se~${A \not\equiv \stfail^i(B)}$ então pela definição de~$\stfail$ como \emph{maior} sufixo de outra classe de equivalência temos que~$|\stfail(\stfail^i(B))| \geq |A|$, uma contradição à maximalidade do~$i$ escolhido. Logo~${\stfail^i(B) \equiv A}$.
\end{proof}

Esses resultados são úteis para entender melhor a relação de equivalência por ocorrências à direita, e serão úteis na criação do algoritmo.
A árvore de sufixos de~$P$ é uma árvore na qual cada vértice representa uma substring de~$P$. No autômato de sufixos, criaremos um digrafo no qual cada vértice representa uma classe de equivalência de ocorrências à direita em~$P$. Assim como na árvore, temos um vértice inicial, e arestas têm caracteres associados, mas no autômato dois caminhos podem levar ao mesmo vértice se as strings associadas a estes forem equivalentes.

\section{Laminaridade}

Seja~$\M{F}$ uma família de subconjuntos de um conjunto~$E$ não-vazio, ou seja, todo elemento de~$\M{F}$ é um subconjunto de~$E$. Dizemos que~$\M{F}$ é laminar se, para quaisquer~$A, B \in \M{F}$, temos que~$A \cap B = \emptyset$ ou~$A \subseteq B$ ou~$B \subseteq A$.

\begin{theorem}
\label{thm:laminar}
Seja~$E$ um conjunto não vazio, e~$\M{F}$ uma família de subconjuntos de~$E$ com~$\emptyset \not\in \M{F}$. Se~$\M{F}$ é laminar então~$|\M{F}| \leq 2|E| - 1$, e se~$E \not\in \M{F}$ então~$|\M{F}| \leq 2|E| - 2$.
\end{theorem}

\begin{proof}
A prova é por indução em~$|E|$. Se~$|E| = 1$, as afirmações valem trivialmente. Suponha que~$|E| > 1$ e que as afirmações valem para toda família laminar de um conjunto de tamanho menor que~$|E|$. Se~$E \in \M{F}$, vamos provar que~$\M{F}' \coloneqq \M{F} \setminus \{E\}$ tem tamanho no máximo~$2|E| - 2$, e então~$|\M{F}| = |\M{F}'| + 1 \leq 2|E| - 1$.

Suponha então que~$E \not\in \M{F}$. Escolha~$A \in \M{F}$ maximal, ou seja, tal que não existe~$B \in \M{F}$ com~$A \subsetneq B$. Considere~$\M{F}_1 = \{X \in \M{F} : X \subseteq A\}$ e~$\M{F}_2 = \{X \in \M{F} : X \cap A = \emptyset\}$. Pela maximalidade de~$A$ e porque~$\M{F}$ é laminar, temos que~$\M{F}_1 \mathbin{\dot{\cup}} \M{F}_2 = \M{F}$. Além disso,~$\M{F}_1$ é laminar sobre o conjunto~$A$ e~$\M{F}_2$ é laminar sobre o conjunto~$E \setminus A$. Ademais,~${1 \leq |A| < |E|}$ e~${1 \leq |E \setminus A| < |E|}$.

Aplicando a hipótese de indução a~$\M{F}_1$ e~$\M{F}_2$, temos que~$|\M{F}_1| \leq 2|A| - 1$ e~${|\M{F}_2| \leq 2|E \setminus A| - 1 = 2|E| - 2|A| - 1}$. Assim~$$|\M{F}| = |\M{F}_1 \mathbin{\dot{\cup}} \M{F}_2| = |\M{F}_1| + |\M{F}_2| \leq 2|A| - 1 + 2|E| - 2|A| - 1 = 2|E| - 2.$$
\end{proof}

\begin{prop}
\label{thm:R3}
A família de conjuntos~$F(P) = \{D(A) : A$ é uma substring de~$P\}$ é laminar, ou seja, se~$D(A) \cap D(B) \neq \emptyset$ então~$D(A) \subseteq D(B)$ ou~$D(B) \subseteq D(A)$.
\end{prop}

\begin{proof}
Seja~$i \in D(A) \cap D(B)$. Então, pela definição,~${A \suff P[1\tdots i]}$ e~${B \suff P[1\tdots i]}$. Logo~${A \suff B}$ ou~${B \suff A}$ e pelo Lema~\ref{lem:R1} segue que~$D(B) \subseteq D(A)$ ou~$D(A) \subseteq D(B)$.
\end{proof}

\begin{corollary} \label{col:oclin}
O tamanho de~$F(P)$ não excede~$2|P|$.
\end{corollary}

\begin{proof}
Como~$D(A) \subseteq \{1, 2, \ldots, |P|\}$ para todo~$D(A) \in F(P)$ tal que~$A \neq \emptystring$,  pelo Teorema~\ref{thm:laminar} vale que~$|F(P) \setminus \{D(\emptystring)\}| \leq 2|P| - 1$ e então~${|F(P)| \leq 2|P|}$.
\end{proof}

Isso mostra que o número de classes de equivalência de~$\equiv_P$ é linear em~$|P|$, mesmo que o número de substrings distintas de~$P$ possa ser quadrático em~$|P|$.

\section{Refinamento das classes de equivalência}
\label{sec:refequiv}

Construiremos o autômato de forma online, similarmente à árvore de sufixos, prefixo por prefixo. Por isso precisamos estudar como a relação de equivalência~$\equiv_P$ muda em relação a~$\equiv_{Pc}$, onde~$c$ é um caractere adicionado ao fim da string~$P$.

Ao adicionar um caractere ao final de~$P$, apenas uma nova ocorrência à direita é possível: a ocorrência na posição~$|P| + 1$. Logo é claro que

\begin{equation}
\label{eq:dext}
D_{Pc}(A) =
\left \{
  \begin{tabular}{ll}
 ~$D_P(A) \cup \{|P| + 1\}$ & se~$A \suff Pc$ \\
 ~$D_P(A)$ & caso contrário.
  \end{tabular}
\right .
\end{equation}

Uma consequência direta disso é a seguinte.

\begin{prop} \label{prop:refin}
Se~$A \equiv_{Pc} B$ então~$A \equiv_P B$, ou seja,~$\equiv_{Pc}$ é um refinamento de~$\equiv_P$.
\end{prop}

\begin{proof}
Se~$A \suff Pc$, então~$D_{Pc}(A) = D_P(A) \cup \{|P| + 1\}$, e assim temos que~${|P| + 1 \in D_{Pc}(A) = D_{Pc}(B)}$. Logo,~$B \suff Pc$ e~$D_{Pc}(B) = D_P(B) \cup \{|P| + 1\}$. Isso implica que~$D_P(A) = D_P(B)$ e então~$A \equiv_P B$.

Analogamente, se~$A \not\suff Pc$, então~${D_{Pc}(A) = D_P(A)}$, e~${|P|+1 \not\in D_{Pc}(A) = D_{Pc}(B)}$, logo~$D_{Pc}(B) = D_P(B)$ e assim~$A \equiv_P B$.
\end{proof}

O lema mostra que, quando adicionamos um caractere ao final de~$P$, as classes de equivalência ou se quebram em classes menores ou se mantém as mesmas, mas nunca se juntam com outras.


Além disso, uma classe nova surge (a classe de equivalência de~$Pc$), e apenas uma classe de equivalência se quebra em classes menores.

\begin{prop} \label{prop:newequiv}
Seja~$X = \stfail_{Pc}(Pc)$. Se~$A \suff Pc$ e~$|A| > |X|$ então~$A \equiv_{Pc} Pc$.
\end{prop}
\begin{proof}
Temos que~$D_{Pc}(Pc) = \{|P| + 1\} = D_{Pc}(A)$, pois pelo Corolário~\ref{col:falha}~$X$ é o \emph{maior} sufixo de~$Pc$ que ocorre em~$P$. Logo~$A$ não ocorre em~$P$.
\end{proof}

\begin{theorem}
\label{thm:refnotX}
Seja~$X = \stfail_{Pc}(Pc)$. Se~$A \equiv_P B$ e~$A \not\equiv_P X$, onde~$A$ e~$B$ são substrings de~$P$, então~$A \equiv_{Pc} B$.
\end{theorem}

\begin{proof}
Vamos mostrar que~$A \suff Pc$ é equivalente a~$B \suff Pc$.
Suponha que~${A \suff Pc}$. Então~${A \suff X}$, pois~$X$ é o maior sufixo de~$Pc$ que ocorre em~$P$ pelo Corolário~\ref{col:falha}.
Pelo Lema~\ref{lem:multifail}, existe~${i \geq 0}$ tal que~${Y = \stfail_P^i(X) \equiv_P A}$. Como~${A \not\equiv_P X}$, temos que~${i > 0}$ e pelo Lema~\ref{lem:R6} vale que~$Y$ é a maior string de sua classe de equivalência. Assim, pelo Corolário~\ref{col:R2}, temos que~${B \suff Y \suffp X \suffp Pc}$.

Como não usamos nenhuma característica especial de~$A$ acima, de maneira semelhante podemos provar que~$B \suff Pc$ implica~$A \suff Pc$. Como~$A \suff Pc$ é equivalente a~$B \suff Pc$, temos por (\ref{eq:dext}) que~$A \equiv_{Pc} B$.
\end{proof}

Esse teorema mostra que, entre as classes de equivalência das substrings de~$P$, apenas a classe de~$X$ pode se dividir, todas as outras continuam as mesmas.


\begin{figure}
\centering
    \begin{tikzpicture}
\matrix[matrix of nodes, nodes in empty cells,
row 1/.style={nodes={draw, minimum size = 5mm}},
row 1 column 23/.style={nodes={draw=none,right}}] (array) {
 & & & & & & & & & & & & & & & & & & & & & &~$Pc$  \\};
 
\draw[decoration={brace, raise=5pt},decorate]
  (array-1-18.north west) -- node[above=6pt] {$X$} (array-1-22.north east);

\draw[decoration={brace, raise=5pt, mirror},decorate]
  (array-1-20.south west) -- node[below=6pt] {$|A| \leq |X|$} (array-1-22.south east);
 
 
\draw[decoration={brace, raise=5pt},decorate]
  (array-1-10.north west) -- node[above=6pt] {$X$} (array-1-14.north east);

\draw[decoration={brace, raise=5pt, mirror},decorate]
  (array-1-5.south west) -- node[below=6pt] {$Y$} (array-1-14.south east);
 
\draw[decoration={brace, raise=20pt},decorate]
  (array-1-8.north west) -- node[above=21pt] {$|A| > |X|$} (array-1-14.north east);
  
 
\end{tikzpicture}

    \caption{Explicação Teorema~\ref{thm:refX}.}
    \label{fig:teo10}
\end{figure}

\begin{theorem}
\label{thm:refX}
Seja~$X = \stfail_{Pc}(Pc)$, e~$Y$ uma string de comprimento máximo tal que~$X \equiv_P Y$. Seja~$A$ uma substring de~$P$ tal que~$A \equiv_P X$, então
\[
A \equiv_{Pc}
\left \{
  \begin{tabular}{ll}
 ~$X$ & se~$|A| \leq |X|$ \\
 ~$Y$ & caso contrário.
  \end{tabular}
\right .
\]
\end{theorem}

\begin{proof}
Pelo Lema~\ref{lem:R6}, temos que~$X$ deve ser o maior de sua classe de equivalência em relação a~$Pc$, mas não a~$P$. Por isso pode ser que~$Y \neq X$. Nesse caso,~$X \suffp Y$ pelo Corolário~\ref{col:R2}. Isso pode ocorrer como na Figura~\ref{fig:teo10}. Note que~$X \suffp Y$ mas~$Y \not\suff Pc$.

Por (\ref{eq:dext}), a classe de equivalência de~$X$ se divide em no máximo duas classes: a dos elementos que são sufixos de~$Pc$ e a dos elementos que não são sufixos de~$Pc$. Se~$|A| \leq |X|$, usando o Corolário~\ref{col:R2}, temos que~$A \suff X \suff Pc$. Logo~$A$ continua na mesma classe que~$X$.
Se~$|A| > |X|$, então~$X \suffp A$ e como~$X$ é o \emph{maior} sufixo de~$Pc$ que ocorre em~$P$, pelo Corolário~\ref{col:falha}, temos que~$A \not\suff Pc$ e também~$Y \not\suff Pc$. Logo ambos~$A$ e~$Y$ continuam na mesma classe de equivalência de~$\equiv_{Pc}$, diferente da classe de~$X$.
\end{proof}

\section{Representação}

A teoria desenvolvida já dá uma ideia de como escrever um algoritmo para construir o autômato. Queremos construir um DAG (Grafo Dirigido Acíclico) em que cada nó represente uma classe de equivalência de~$\equiv_P$ (ou seja, um conjunto de ocorrências à direita), e uma aresta indique, ao adicionar um caractere ao final de uma string da classe de equivalência do nó cabeça da aresta, para qual classe de equivalência esta string muda. A Proposição~\ref{prop:classc} mostra que a classe de equivalência ao adicionar um caractere não depende da string escolhida.

O DAG terá um nó inicial (chamaremos de raiz) que indica o nó relativo à classe de equivalência de~$\emptystring$. Um caminho a partir da raiz representa uma substring de~$P$, e note que vários caminhos podem levar ao mesmo nó.

A Seção~\ref{sec:refequiv} mostra como atualizar esses nós ao adicionar um caractere a~$P$. A Proposição~\ref{prop:newequiv} mostra quais novas substrings existem, que tem a mesma classe que a string~$Pc$. O Teorema~\ref{thm:refnotX} prova que apenas a classe de~${X = \stfail_{Pc}(Pc)}$ muda, podendo se particionar em duas. Esse particionamento é então detalhado no Teorema~\ref{thm:refX}.

Seja~$u$ uma classe de equivalência de~$\equiv_P$. Usaremos que~$u$ é um nó do autômato, e que~$u.f$ é a lista de adjacências de~$u$, ou seja, guarda todas as arestas que saem de~$u$, indexadas por seu caractere. Assim como anteriormente,~$u.f$ é implementada usando um vetor de~$|\E|$ posições. Diferentemente da árvore de sufixos, cada aresta representa apenas um caractere, logo a implementação dessa lista é mais simples. Usamos que~$S(u)$ é a maior string da classe~$u$.

Usaremos que~$u.\stfail$ é a classe de equivalência da string de falha de alguma string da classe~$u$. Note que, pelo Lema~\ref{lem:R4}, esse valor independe da string escolhida na classe~$u$, logo está bem definido. Usaremos que~$u.L$ é o tamanho da maior string em~$u$, ou seja, é~$|S(u)|$. Além disso, usamos~$\textbf{new } \node(x)$ para representar a criação de um nó com campo~$L$ inicializado com~$x$, lista de adjacências vazia e campo~$\stfail$ nulo.

O autômato será construído adicionando um caractere por vez ao final da string. Assim, ao construir o autômato de~$Pc$, é necessário atualizar as falhas~$\stfail_P$ para~$\stfail_{Pc}$, e o seguinte lema ajuda nisso.

\begin{lemma} \label{lem:failpc}
Se~$A \suff P$ e~$\stfail_{Pc}(Ac) \neq \emptystring$, então~${\stfail_{Pc}(Ac) = \stfail_P^i(A)c}$ para algum~${i > 0}$ e, para todo~$j < i$, vale que~$\stfail_P^j(A)c \equiv_P Ac$.
\end{lemma}

\begin{proof}
Seja~${Bc = \stfail_{Pc}(Ac)}$. Pela definição de~$\stfail$, temos que~${B \suff A}$. Suponha que exista~${Z \equiv_P B}$ tal que~$B \suffp Z$. Então, pela Proposição~\ref{prop:classc}, vale que~${Bc \equiv_P Zc}$, e como ambos~$B$ e~$Z$ são sufixos de~$P$, vale que~${\stfail_{Pc}(Ac) \equiv_{Pc} Zc}$ e~${|Zc| > |\stfail_{Pc}(Ac)|}$, uma contradição pelo Lema~\ref{lem:R6}.
Logo~$B = \stfail_P^i(A)$ para algum~$i > 0$ (caso contrário,~$\stfail$ ``pularia'' uma classe de equivalência), e assim~$\stfail_{Pc}(Ac) = \stfail_P^i(A)c$.

Pela definição de~$\stfail$, para todo~$j < i$, vale~${\stfail_P^j(A)c \equiv_{Pc} Ac}$ e, usando a Proposição~\ref{prop:refin}, temos~${\stfail_P^j(A)c \equiv_P Ac}$.
\end{proof}

\section{Implementação}

O algoritmo no Código~\ref{lst:suffauto} constrói o autômato de sufixos para uma string~$P$, de acordo com o que foi discutido. A Figura~\ref{fig:bananaauto} ilustra como a construção funciona para a string~\texttt{banana}.

\begin{algorithm}
\caption{Autômato de sufixos em tempo linear}
\label{lst:suffauto}
\begin{algorithmic}[1]
\Function{BuildSuffixAutomaton}{$P$}
    \State $\last = \root = \New \node(0)$ \label{auto:init}
    \For{$i = 1 \To |P|$} \label{auto:for}
        \State $p = \last$
        \State $\last = \New \node(i)$ \Comment{classe da string~$P[1\tdots i]$}
        \While{$p \neq \Null \And p.f[P[i]] = \Null$} \label{auto:while1s}
            \State $p.f[P[i]] = \last$ \label{auto:newclass}
            \State $p = p.\stfail$ \label{auto:while1e}
        \EndWhile
        \If{$p = \Null$}
            \State $\last.\stfail = \root$ \label{auto:caso1}
        \Else
            \State $y = p.f[P[i]]$ \Comment{$X$ é a string~$S(p)P[i]$} \label{auto:caso23} \label{auto:findX}
            \If{$y.L = p.L + 1$} \Comment{se~$X = S(y)$} \label{auto:caso2s}
                \State $\last.\stfail = y$ \label{auto:caso2e}
            \Else \Comment{a classe de~$X$ se divide em duas}
                \State $x = \New \node(p.L + 1)$ \label{auto:caso3s}
                \State $x.f = y.f$ \label{auto:copyf}
                \State $x.\stfail = y.\stfail$
                \State $y.\stfail = \last.\stfail = x$ \label{auto:failupd}
                \While{$p \neq \Null \And p.f[P[i]] = y$} \label{auto:while2s}
                    \State $p.f[P[i]] = x$ \Comment{Trocando adjacência de~$y$ para~$x$}
                    \State $p = p.\stfail$ \label{auto:caso3e} \label{auto:while2e}
                \EndWhile
            \EndIf
        \EndIf
    \EndFor
    \State $\Return$~$\root$
\EndFunction
\end{algorithmic}
\end{algorithm}


%%%%%%%%%%%%%%%%%%%%%%%%%%%%%%%%%%%%%%%%%
%%%%%% CONSTRUCAO BANANA %%%%%%%%%%%%%%%%
%%%%%%%%%%%%%%%%%%%%%%%%%%%%%%%%%%%%%%%%%
\begin{figure}
\subfloat[$\emptystring$] {
\begin{minipage}{0.5\textwidth}
\centering
\hspace{-1.6cm}
\begin{tikzpicture}[->,>=stealth',shorten >=1pt,node distance=2cm, auto, initial text={root}]
\node[initial, state] (0) {};

\end{tikzpicture}
\end{minipage}
}
\hfill
\subfloat[b] {
\begin{minipage}{0.5\textwidth}
\hspace{-1.5cm}
\centering
\begin{tikzpicture}[->,>=stealth',shorten >=1pt,node distance=2cm, auto, initial text={root}]
\node[initial, state] (0) {};
\node[state] (1) [right of=0] {};


\path (0) edge              node {b} (1);

\end{tikzpicture}
\end{minipage}
}
\hfill
\subfloat[ba] {
\begin{minipage}{0.35\textwidth}
\centering
\begin{tikzpicture}[->,>=stealth',shorten >=1pt,node distance=2cm, auto, initial text={root}]
\node[initial, state] (0) {};
\node[state] (1) [right of=0] {};
\node[state] (2) [right of=1] {};


\path (0) edge              node {b} (1)
          edge [bend left]  node {a} (2)
      (1) edge              node {a} (2);

\end{tikzpicture}
\end{minipage}
}
\hfill
\subfloat[ban] {
\begin{minipage}{0.5\textwidth}
\centering
\begin{tikzpicture}[->,>=stealth',shorten >=1pt,node distance=2cm, auto, initial text={root}]
\node[initial, state] (0) {};
\node[state] (1) [right of=0] {};
\node[state] (2) [right of=1] {};
\node[state] (3) [right of=2] {};


\path (0) edge              node {b} (1)
          edge [bend left]  node {a} (2)
          edge [bend right] node {n} (3)
      (1) edge              node {a} (2)
      (2) edge              node {n} (3);

\end{tikzpicture}
\end{minipage}
}
\hfill
\subfloat[bana] {
\begin{minipage}{\textwidth}
\centering
\begin{tikzpicture}[->,>=stealth',shorten >=1pt,node distance=2cm, auto, initial text={root}]
\node[initial, state] (0) {};
\node[state] (1) [right of=0] {};
\node[state] (2) [right of=1] {};
\node[state] (3) [right of=2] {};
\node[state] (4) [right of=3] {};
\node[state] (7) [below of=2] {};


\path (0) edge              node {b} (1)
          edge [bend left]  node {n} (3)
          edge [bend right] node {a} (7)
      (1) edge              node {a} (2)
      (2) edge              node {n} (3)
      (3) edge              node {a} (4)
      (7) edge [bend right] node {n} (3);

\path (2) edge [dotted,>=latex']             node {} (7)
      (4) edge [dotted,>=latex',bend left]   node {} (7.-25);

\end{tikzpicture}
\end{minipage}
}
\hfill
\subfloat[banan] {
\begin{minipage}{\textwidth}
\centering
\begin{tikzpicture}[->,>=stealth',shorten >=1pt,node distance=2cm, auto, initial text={root}]
\node[initial, state] (0) {};
\node[state] (1) [right of=0] {};
\node[state] (2) [right of=1] {};
\node[state] (3) [right of=2] {};
\node[state] (4) [right of=3] {};
\node[state] (5) [right of=4] {};
\node[state] (7) [below of=2] {};
\node[state] (8) [below of=3] {};


\path (0) edge              node {b} (1)
          edge [bend right] node {a} (7)
      (0.-30) edge          node {n} (8.130)
      (1) edge              node {a} (2)
      (2) edge              node {n} (3)
      (3) edge              node {a} (4)
      (4) edge              node {n} (5)
      (7) edge              node[below] {n} (8)
      (8) edge              node {a} (4);

\path (2) edge [dotted,>=latex']             node {} (7)
      (3) edge [dotted,>=latex']             node {} (8)
      (5) edge [dotted,>=latex',bend left]   node {} (8)
      (4) edge [dotted,>=latex']             node {} (7);

\end{tikzpicture}
\end{minipage}
}
\hfill
\subfloat[banana]{
\begin{minipage}{\textwidth}
\centering
\begin{tikzpicture}[->,>=stealth',shorten >=1pt,node distance=2cm, auto, initial text={root}]
\node[initial, state] (0) {};
\node[state] (1) [right of=0] {};
\node[state] (2) [right of=1] {};
\node[state] (3) [right of=2] {};
\node[state] (4) [right of=3] {};
\node[state] (5) [right of=4] {};
\node[state] (6) [right of=5] {};
\node[state] (7) [below of=2] {};
\node[state] (8) [below of=3] {};
\node[state] (9) [below of=4] {};


\path (0) edge              node {b} (1)
      (0) edge [bend right] node {a} (7)
      (0.-30) edge          node {n} (8.130)
      (1) edge              node {a} (2)
      (2) edge              node {n} (3)
      (3) edge              node {a} (4)
      (4) edge              node {n} (5)
      (5) edge              node {a} (6)
      (7) edge              node[below] {n} (8)
      (8) edge              node {a} (9)
      (9) edge              node {n} (5);

\path (2) edge [dotted,>=latex']             node {} (7)
      (3) edge [dotted,>=latex']             node {} (8)
      (4) edge [dotted,>=latex']             node {} (9)
      (5) edge [dotted,>=latex']             node {} (8)
      (6) edge [dotted,>=latex',bend left]   node {} (9)
      (9) edge [dotted,>=latex',bend left]   node {} (7);

\end{tikzpicture}
\end{minipage}
}

\caption{Construção do autômato de sufixos de \texttt{banana}, falhas desenhadas pontilhadas. As falhas para a raiz não estão desenhadas para não obstruir a imagem.}
\label{fig:bananaauto}
\end{figure}
%%%%%%%%%%%%%%%%%%%%%%%%%%%%%%%%%%%%%%%%%
%%%%%% FIM CONSTRUCAO BANANA %%%%%%%%%%%%
%%%%%%%%%%%%%%%%%%%%%%%%%%%%%%%%%%%%%%%%%



\begin{invar}
No início da~$i$-ésima iteração do \keyword{for} da linha~\nref{auto:for},~$\root$ é a raiz do autômato de~$P[1\tdots i-1]$ e~$\last$ é o nó que representa a classe de~$P[1\tdots i-1]$ nesse autômato.
\end{invar}

\begin{proof}
Na primeira iteração do \keyword{for}, o autômato está corretamente calculado, pois o autômato de uma string vazia é apenas um nó (associado à classe da string~$\emptystring$), e a linha~\nref{auto:init} inicializa corretamente os campos~$\root$ e~$\last$.

No começo de uma iteração, temos o autômato de~$P[1\tdots i-1]$ e queremos calcular o autômato de~$P[1\tdots i]$, para que o invariante continue valendo. Para continuar com a mesma notação de antes, vamos usar que temos o autômato de~${Q = P[1\tdots i-1]}$ e queremos calcular o autômato de~$Qc$, onde~${c = P[i]}$.

Uma nova classe de equivalência sempre se forma, a de~$Qc$, e de acordo com a Proposição~\ref{prop:newequiv}, todos os sufixos de~$Qc$ maiores que~$X = \stfail_{Qc}(Qc)$ estão nessa classe de equivalência. A variável~$\last$ é atualizada para o novo nó que representará essa classe. A variável~$p$ passa a guardar o valor antigo de~$\last$, e será utilizada para iterar pelas classes dos sufixos de~$Q$.

O \keyword{while} nas linhas~\nref{auto:while1s}-\nref{auto:while1e} itera pelas classes de equivalência dos sufixos de~$Q$ enquanto~${p.f[c] = \keyword{null}}$, ou seja, enquanto~${S(p)c \equiv_Q Qc}$ (pois~${D_Q(Qc) = \emptyset}$). Vale então, pelo Lema~\ref{lem:failpc}, que após esse \keyword{while}, se temos~$p \neq \keyword{null}$ então~${S(p)c = X}$, e assim o nó~$p.f[c]$ é a classe de equivalência de~$X$. Se~${p = \keyword{null}}$, então o caractere~$c$ não ocorre em~$Q$ e a linha~\nref{auto:caso1} corretamente atualiza a falha~$\stfail$ de~$\last$ para indicar a raiz, pois nesse caso~${\stfail_{Qc}(Qc) = \emptystring}$.
Note que a linha~\nref{auto:newclass} atualiza a lista de adjacências dos nós acessados por~$p$, e está correta pela Proposição~\ref{prop:newequiv}.

Se a linha~\nref{auto:caso23} é atingida, temos que~$y$ é a classe de equivalência da string~$X$. Então, pelo Teorema~\ref{thm:refX}, essa classe pode se quebrar em duas classes de equivalência. As linhas~${\nref{auto:caso2s}\text{-}\nref{auto:caso2e}}$ tratam o caso em que~$X = S(y)$, ou seja, no teorema temos~${X = Y}$ e a classe de equivalência não se separa.
As linhas~\nref{auto:caso3s}-\nref{auto:caso3e} tratam o caso em que~$S(y) = Y \neq X$. Nesse caso, é necessário quebrar a classe em duas, e o novo nó criado~$x$ vai ser a classe das strings em~$y$ com tamanho que não excede~$|X| = p.L + 1$.
A linha~\nref{auto:copyf} copia a lista de adjacências de~$y$ para~$x$, já que as adjacências continuam iguais, e a linha seguinte copia a falha~$\stfail$ de~$y$ para~$x$.
A linha~\nref{auto:failupd} atualiza as falhas de~$\last$ e~$y$, pois ambas devem ser~$x$. De fato,~$\last.\stfail = x$ pois a string~${X = \stfail_{Qc}(Qc)}$ está em~$x$ e~$y.\stfail = x$ pois~$X \suffp Y$ e tem tamanho máximo, já que antes~$Y$ e~$X$ estavam na mesma classe.

Resta atualizar todo nó~$u$ tal que~$u.f[c] = y$ e~$|S(u)| \leq |X|$ para~$u.f[c] = x$, então pelo Teorema~\ref{thm:refX} essas classes estarão corretas. Note que, para tais nós~$u$, teremos~${S(u)c \suff X = S(p)c}$, e assim podemos acessá-los pelas falhas de~$p$.
O~\keyword{while} das linhas~\nref{auto:while2s}-\nref{auto:while2e} atualiza as listas de adjacências necessárias, pois iteramos enquanto~${p.f[c] = y}$, ou seja, enquanto~${S(p)c \equiv_Q X}$.

Pelo Teorema~\ref{thm:refnotX}, todas as outras classes de equivalências continuam as mesmas, e assim o autômato de~$Qc = P[1\tdots i]$ é corretamente calculado, mantendo o invariante.
\end{proof}

Pelo invariante, temos que ao final do algoritmo o autômato de sufixos de~$P$ está corretamente calculado.
Provar a complexidade, contudo, não é tão simples. O Corolário~\ref{col:oclin} mostra que o número de nós do autômato é~$\Oh(|P|)$, mas ainda temos que o número de arestas pode ser~$\Theta(|P||\E|)$, se de cada vértice sair uma aresta para cada vértice, por exemplo. Esse, porém, não é caso, como mostraremos abaixo.

Dizemos que um nó é final se ele representa uma classe de equivalência de um sufixo de~$P$. Um caminho da raiz à um nó final indica um sufixo de~$P$. A partir de um nó não-final, é sempre possível encontrar um caminho a um nó final, pois é sempre possível completar uma substring de~$P$ para que esta se torne um sufixo. Essas noções serão úteis na prova da proposição abaixo.

\begin{prop} \label{prop:arlin}
O número de arestas do autômato de sufixos de~$P$ não excede~$3|P| - 1$.
\end{prop}

\begin{proof}
%Note que para cada nó, não existem dois caminhos de mesmo tamanho da raiz até este, pelo Lema~\ref{col:R2}. Assim, existe apenas uma aresta de caminho máximo que chega neste. Se escolhemos essas arestas para todos os nós diferentes da raiz temos então uma árvore geradora do autômato. Seja~$T$ essa árvore.

Como a partir da raiz é possível atingir qualquer vértice, seja~$T$ uma árvore geradora qualquer do autômato.

Para cada aresta~${e = uv}$ que não pertence a~$T$, considere a string~$AcB$, onde~$A$ é a string associada ao caminho da raiz a~$u$ na árvore~$T$,~$c$ é o caractere associado à aresta~$e$, e~$B$ é a string associada ao caminho de~$v$ a algum vértice final (tal caminho sempre existe pois é possível completar uma substring para um sufixo).

Todas essas strings são distintas, já que cada uma tem a primeira aresta que não está em~$T$ diferente. Além disso, como acabam em um vértice final, temos que~${AcB \suff P}$, para toda~${e \not\in T}$. Logo, existem no máximo~$|P|$ arestas fora da árvore, pois~$P$ tem~$|P|$ sufixos. Pelo Corolário~\ref{col:oclin}, temos que o número de arestas em~$T$ é no máximo~${2|P| - 1}$, e assim o número de arestas do autômato é no máximo~${3|P| - 1}$.
\end{proof}

\begin{complexity}
O algoritmo no Código~\ref{lst:suffauto} consome tempo~$\Theta(|P||\E|)$.
\end{complexity}

\begin{proof}

Como o~\keyword{for} da linha~\nref{auto:for} tem exatamente~$|P|$ iterações, todas as operações neste que consomem tempo constante (atribuições, somas, etc.) vão no total consumir tempo~$\Oh(|P|)$. Se consideramos que as operações~${\textbf{new } \node}$ consomem tempo~$\Theta(|\E|)$, então no total elas consumirão tempo~$\Theta(|P||\E|)$. Resta analisar os~\keyword{while}s das linhas~\nref{auto:while1s} e~\nref{auto:while2s}, e a instrução da linha~\nref{auto:copyf}, pois esta não consome tempo constante.

A instrução de copiar a lista de adjacências na linha~\nref{auto:copyf} precisa copiar cada aresta de uma lista de adjacências para outra. Como estamos considerando a implementação dessa lista usando um vetor, é necessário copiar todas as~$|\E|$ posições de um vetor para outro, então no total isso consome tempo~$\Oh(|P||\E|)$.

Como já foi provado pelo Corolário~\ref{col:oclin} e pela Proposição~\ref{prop:arlin}, o número de vértices e arestas do autômato final é~$\Oh(|P|)$, e note que o número de vértices e arestas nunca diminui durante o algoritmo. Logo, as operações que ocorrem apenas uma vez por criação de nó ou aresta levam tempo linear.
Isso mostra que o~\keyword{while} da linha~\nref{auto:while1s} consome tempo linear, pois cada iteração deste cria uma aresta.
Resta então provar a complexidade do~\keyword{while} das linhas~\nref{auto:while2s}-\nref{auto:while2e}, pois nessas linhas estamos ``trocando'' arestas, então isso potencialmente poderia ocorrer muitas vezes.

Considere novamente que na~$i$-ésima iteração temos o autômato de sufixos da string~${Q = P[1\tdots i-1]}$ e iremos adicionar o caractere~${c = P[i]}$. Na linha~\nref{auto:findX}, temos que~$y$ é a classe de equivalência de~$X = \stfail_{Qc}(Qc)$ na string~$Q$, então considere o Lema~\ref{lem:failpc}, onde~${Ac = X}$. Temos que~$p$ é a classe de equivalência de~$A$. Note que o~\keyword{while} itera pelas falhas de~$p$ enquanto~${p.f[c] = y}$, ou seja, enquanto~$S(p) \equiv_Q X$. Pelo Lema~\ref{lem:failpc}, temos que, quando esse laço acaba,~$p.f[c]$ é a classe de equivalência de~$\stfail_{Qc}(X)$ na string~$Qc$, ou seja, a classe de~$\stfail^2_{Qc}(Qc)$. Note que~${\stfail^2_{Qc}(Qc) = S(p)c}$, ou seja,~${|\stfail^2_{Qc}(Qc)| = p.L + 1}$.

Agora vamos novamente usar um argumento similar ao da prova da complexidade do Aho-Corasick e da árvore de sufixos. Vamos observar o valor de~$p.L$ durante o algoritmo, ou seja, o tamanho da maior string de~$p$. Note que a operação~${p = p.\stfail}$ sempre diminui~$p.L$ em pelo menos~$1$. Em cada iteração,~$p$ é inicializado como~$\last$, ou seja, inicialmente~${p.L = i - 1}$. Perceba que as condições dos~\keyword{while}s das linhas~\nref{auto:while1s} e~\nref{auto:while2s} são sempre satisfeitas pelo menos uma vez (de fato, poderíamos ter usado~\keyword{do-while}), logo ao final da iteração temos~${p.L \leq |\stfail^2_{Q}(Q)|}$.

Usando o que foi discutido anteriormente, temos~${|\stfail^2_{Qc}(Qc)| = p.L + 1 \leq |\stfail^2_{Q}(Q)| + 1}$. Assim vale que o valor de~$|\stfail^2_{P[1\tdots i]}(P[1\tdots i])|$ sempre aumenta em no máximo~$1$ entre iterações, e cada iteração do~\keyword{while} da linha~\nref{auto:while1s} ou da linha~\nref{auto:while2s} diminui esse valor em pelo menos~$1$. Seja~$w_i$ o número de iterações além da primeira dos dois~\keyword{while}s. Vale que~${w_i \leq (|\stfail^2_{P[1\tdots i-1]}(P[1\tdots i-1])| + 1) - |\stfail^2_{P[1\tdots i]}(P[1\tdots i])|}$. Para que essa notação funcione mesmo quando~${i = 1}$, vamos usar que~${\stfail_\emptystring(\emptystring) = \emptystring}$. Temos então que
\begin{equation*} \begin{split}
\sum\limits_{i=1}^{|P|}{w_i} & \leq \sum\limits_{i=1}^{|P|}{(|\stfail^2_{P[1\tdots i-1]}(P[1\tdots i-1])| - |\stfail^2_{P[1\tdots i]}(P[1\tdots i])| + 1)} \\
                         & = |\stfail^2_\emptystring(\emptystring)| - |\stfail^2_P(P)| + |P| \quad\text{(soma telescópica)}\\
                         & \leq |P| \quad\text{(pois~$|\stfail^2_\emptystring(\emptystring)| = 0$ e ~$|\stfail^2_P(P)| \geq 0$).}
\end{split}
\end{equation*}

Isso prova que os \keyword{while}s levam tempo linear e assim o algoritmo tem complexidade de tempo~$\Theta(|P||\E|)$.

\end{proof}

É fácil ver que o espaço da estrutura criada também é~$\Theta(|P||\E|)$. Usando a implementação de listas de adjacências como ABBBs, usando o caractere de uma aresta como chave, e sua ponta final como valor, é possível provar de forma muito semelhante que o algoritmo consome tempo~$\Theta(|P|\log|\E|)$ e espaço~$\Theta(|P|)$.

\section{Uso}

Com o autômato de~$P$ construído, é simples saber se uma string~$S$ é substring de~$P$. Basta verificar se existe caminho~${S[1], S[2], \ldots, S[|S|]}$ no autômato, a partir da raiz.

Contar o número de ocorrências ou enumerá-las não é tão simples. É necessário descobrir quais são os nós finais do nosso autômato.

\begin{prop}
Um nó do autômato de~$P$ é final se e somente se pertence ao conjunto~$\Path(\last) = \{\last, \last.\stfail, \last.\stfail.\stfail, \ldots, \root\}$.
\end{prop}

\begin{proof}
O nó~$\last$ claramente é final, pois é a classe de equivalência de~$P$. Assim, como~${S(u.\stfail) \suffp S(u)}$ para qualquer~${u \neq root}$, pela definição de~$\stfail$, temos que qualquer elemento do conjunto~$\Path(\last)$ é um nó final. Por outro lado, se~${X \suff P}$, pelo Lema~\ref{lem:multifail}, existe~${i \geq 0}$ tal que~${\stfail^i(P) \equiv X}$, e portanto a classe de equivalência de~$X$ está em~$\Path(\last)$.
\end{proof}

Portanto, se adicionarmos um campo booleano~$\final$ à nossa implementação de nós, que é inicializado com~$\keyword{false}$, e indica se um nó é final ou não, o seguinte código pode ser adicionado ao final da criação do autômato de sufixos para marcar os nós finais corretamente.

\begin{algorithm}
\begin{algorithmic}[1]
\State $p = \last$
\While{$p \neq \Null$}
    \State $p.\final = \True$
    \State $p = p.\stfail$
\EndWhile
\end{algorithmic}
\end{algorithm}

Agora, para contar o número de ocorrências de alguma string~$S$ em~$P$, percorremos o caminho~${S[1], \ldots, S[|S|]}$ no autômato de sufixos de~$P$. Seja~$v$ o nó alcançado. O número de ocorrências de~$S$ em~$P$ é o número de caminhos distintos de~$v$ para um nó final, pois este é o número de formas de completar a string~$S$ para ser um sufixo de~$P$. Isso é similar a como, na árvore de sufixos, o número de ocorrências de uma substring representada por um nó é o número de folhas na subárvore deste.

Como o autômato de sufixos é um DAG, ele tem uma ordenação topológica, e por isso calcular o número de caminhos a partir de cada vértice para algum nó final é simples. Para calcular uma ordenação topológica de um DAG, basta realizar uma DFS e armazenar os vértices por ordem inversa de finalização da DFS, como discutido por Cormen et al.~\cite[Seção 22.4]{cormen}.
\newcommand{\Count}{\textsc{Count}}
Dado um vetor~$\top$ representando uma ordenação topológica do autômato, ou seja, se~$\top[i] \top[j]$ é uma aresta, então~${i < j}$, é possível calcular o número de caminhos de~$u$ para algum vértice final ($\Count(u)$) usando a seguinte recorrência
\[
\Count(u) =
\left \{
  \begin{tabular}{ll}
 ~$\sum\limits_{v \in u.f}{\Count(v)}$ & se~$u$ não é final \\
 ~$1 + \sum\limits_{v \in u.f}{\Count(v)}$ & se~$u$ é final.
  \end{tabular}
\right .
\]
Ou seja, o número de caminhos de~$u$ para algum vértice final é o número de caminhos de seus vizinhos para algum vértice final (tratando o caso quando esses caminhos tem comprimento maior que~$0$), somado de~$1$ se~$u$ for final (tratando o caso quando o comprimento é~$0$). A recorrência está correta pois o digrafo é acíclico.

O seguinte código determina o valor de~$\Count$ para cada vértice~$u$ em~$C[u]$.

\begin{algorithm}
\begin{algorithmic}[1]
\For{$i = |\top| \DownTo 1$}
    \State $C[\top[i]] = 0$
    \For{$v \in \top[i].f$}
        \State $C[\top[i]] \pluseq C[v]$
    \EndFor
    \If{$\top[i].\final$}
        \State $C[\top[i]] \pluseq 1$
    \EndIf
\EndFor
\end{algorithmic}
\end{algorithm}

Se temos como invariante que, ao analisar o vértice~$\top[i]$, todos os valores de~$C$ para vértices~$\top[j]$, com~${j > i}$, já estão calculados, então, como~$\top$ é uma ordenação topológica, todos os vértices adjacentes a~$\top[i]$ tem o valor correto, logo o valor~$C[\top[i]]$ é calculado corretamente. A base é trivial quando~$i = |\top|$.

Logo, este código funciona e consome tempo proporcional ao número de vértices mais arestas do autômato, que é linear em~$|S|$, pelo Corolário~\ref{col:oclin} e pela Proposição~\ref{prop:arlin}.