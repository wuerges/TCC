\chapter*{Parte Subjetiva}
\chaptermark{Parte Subjetiva}
\addcontentsline{toc}{chapter}{Parte Subjetiva}

\setcounter{secnumdepth}{0}

\section{O processo, desafios e frustrações}

Escolhi o tema pois durante a minha graduação me identifiquei com a área de teoria, porém com um foco em implementação. Isso se deve em grande parte por praticar programação competitiva, participando da Maratona de Programação.
O tema em strings, e não grafos, programação dinâmica ou algum outro tópico bem visto em programação competitiva se deve ao fato que eu tinha pouca familiaridade com strings, e quis então aprender e me especializar nessa área.

Decidi estudar algoritmos complicados como árvore e autômato de sufixos, que poucas pessoas mesmo em programação competitiva conhecem, pelo desafio de conseguir implementar uma estrutura complicada de forma eficiente.
E foi um grande desafio, pois muitas vezes, ao estudar esses temas, eu encontrava grande dificuldade em entender os conceitos, e não conseguia continuar, parando por dias, ou até semanas, até voltar ao assunto.

Não tinha passado muitas vezes por isso, e aprendi que esse é um passo importante para aprender um tópico difícil. Mesmo desistindo do assunto por algum tempo, sempre voltava ao assunto com uma base mais forte, e conseguia, depois de duas, três, ou mais tentativas, finalmente entender o tópico em suas entranhas, e saber por que cada parte funciona. Esse sentimento é muito gratificante, depois de semanas de frustração.

\section{Graduação e o trabalho de formatura}

As matérias <<Introdução à Teoria dos Grafos>> e <<Otimização Combinatória>> me ensinaram a provar afirmações e pensar formalmente, com atenção a afirmações vazias ou confusões comuns nas provas.

As matérias <<Algoritmos em Grafos>> e <<Análise de Algoritmos>> também me ensinaram essas questões, em menor grau, mas com grande atenção à implementação e como provar corretude e complexidade de código.

Praticar programação competitiva, e também a matéria <<Desafios de Programação>>, me fizeram entender claramente o meu código, e conseguir implementar ideias complicadas de forma muito mais simples do que eu já teria imaginado.

\section{Futuro}

Nessa área, tenho interesse em explorar um pouco mais as estruturas desenvolvidas no texto. A árvore de sufixos, por exemplo, é simples de ser implementada para armazenar múltiplas strings, e já resolvi diversos problemas com esta. O autômato de sufixos, entretanto, não consegui desenvolver para múltiplas strings, e nem resolver algum problema de programação competitiva que não havia resolvido com árvore de sufixos.

Outro tópico interessante de se estudar são as similaridades entre árvore e autômato de sufixos, pois é possível construir um a partir do outro de forma muito mais simples que construir do início. Acredito que estudar isso daria um entendimento maior das duas estruturas.

\section{Agradecimentos}

Gostaria de agradecer a prof. Cristina, por aceitar minhas ideias de tema e me apoiar no estudo dessas estruturas, sempre tendo tempo para rabiscar meus rascunhos e me ajudando a escrever um texto melhor!

Agradeço também a todos meus amigos que praticam programação competitiva comigo, em especial no grupo MaratonIME, pois a motivação destes me fez começar a estudar bastante nessa área e me mantém focado nisso até hoje!